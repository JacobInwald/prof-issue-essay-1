\documentclass{article}
\usepackage[backend=biber,style=numeric,autocite=plain,sorting=none,maxbibnames=99]{biblatex}
\usepackage{graphicx} % Required for including pictures
\graphicspath{ {figures} }
\usepackage{hyperref} % Format links for pdf
\usepackage[none]{hyphenat}
\addbibresource{essay.bib}
\frenchspacing
\usepackage{float}
\usepackage{wrapfig}
\usepackage{caption}
\usepackage{subcaption}
\usepackage[margin=1in]{geometry}


\begin{document}
The system chosen for review is a drone delivery service which delivers pizza to Appleton Tower. 
This essay accepts the definition of harm given by the ACM code of ethics - ``harm means negative consequences, especially when those consequences are significant and unjust"\supercite{ACM-ethics}. 
This definition includes but is not limited to physical and non-physical injury, misuse of personal data, and environmental damage. 
The system will not hold any user data - card details will not be saved to a server - so misuse of personal data is not a potential harm here.
\newline 
\newline 
A clear potential for harm in the system is the possibility of physical injury, which is my responsibility as the developer to mitigate.
This could be caused by violating no-fly zones, causing injury via disruption\supercite{guardian2018airportdrone}, or by malfunction over people\supercite{drone-harm}. 
According to the UK Civil Aviation Authority (CAA), no-fly zones for drones limit their height to be under 120m and include any column of air above a group of people\supercite{CAA-drone-code}.  
Therefore, as long as the drone routes over rooftops that are not populated, and flies under 120m, harm due to violating no-fly zones will be mitigated. 
However, this raises environmental concerns.
Mulero-Pázmány et al.\supercite{drone-wildlife-review} found that the noise of the drone harms animals as they panic and injure themselves, with the most intense reactions found in birds. 
As the drone will now fly over rooftops to comply with no-fly zones, its noise will harm the birds roosting there. 
To prevent harm due to noise, I should ensure that my colleagues use a ``quiet" propeller\supercite{toroidal-props} when designing the drone, however, it will not be possible to fully mitigate harm such as collisions with wildlife. 
\newline 
\newline 
Additionally, there are non-physical causes of harm. 
This system is funded by the University of Edinburgh(UoE), raising concerns over the fairness of the system. 
According to the specification, the system only delivers to Appleton Tower\supercite{pizza-dronz-spec}.
Appleton Tower is central to the Informatics School\supercite{Appleton-Tower-History}, meaning users will predominantly be informatics students. 
From 2021-2022 there were a total of 49,065 students at UoE - and 2,305 were informatics students\supercite{UoE-stats}, comprising a small percentage of the student base($\sim4.7\%$). 
This is unfair to the non-informatics students as just $\sim4.7\%$ of students will use this service, even though the tuition fees of all students fund it. 
A valid counterargument is that there is precedent for these kinds of investments, in 2023 UoE invested over £150m in direct holdings alone\supercite{edinburgh-investments}, and these investments ultimately help the whole student base as they provide more money to UoE to be spent on the student experience. 
However, the fact remains that this system will not be accessible by all, compounded by the risk it brings of not turning a profit, and wasting money that could otherwise be spent on directly improving student experience. 
UoE will need to be held accountable on whether funding this system is fair or not.
\newline 
\newline
Another potential for harm is the problem of automation.
According to the Office of National Statistics, $60\%$ of delivery jobs are at risk of automation\supercite{ONS-automation}, and systems like this one are the most likely candidates to automate those jobs. 
Patel et al. (2018) found an association between the risk of job loss due to automation and poor health, both mental and physical.\supercite{PATEL201854}
This demonstrates that job loss due to this system will cause harm to the workers it replaces.
Moreover, as Berkowitz (2014) points out, automation also removes the human link (the presence of humans in a decision chain) in favour of more efficient functionality\supercite{berkowitz_2014}. 
This human link is essential to preventing the propagation of mistakes through an automated decision chain\supercite{berkowitz_2014}. 
These mistakes are normally obvious to a human but hard for a machine to understand. 
One such mistake could be an incorrect pizza price: a human delivery driver would understand that pizzas are not 1 pence and therefore report the error; a drone would miss this and therefore deliver the order. 
In this case, automation would cause the restaurant harm by undercharging their customers. 
To prevent these mistakes the system should have an administrator who ensures that the data in the system is valid. 
Currently, the scope of the system is too small to replace jobs as the drone can only carry out 30-40 deliveries a day. 
However, both of these problems will be important to review when scaling the system up, and UoE is responsible for ensuring these concerns are addressed. 
\newline
\newline
Having discussed some potential causes for harm, the role and responsibility of the stakeholders in question need to be discussed. 
As the developer, my role is to create and maintain the system. 
My responsibility is to mitigate the harm which is caused directly by the system or by flaws in its functionality.
Additionally I should bring up any concerns with management and then resolve according to advice.
As the main stakeholders, UoE's role is to oversee the project and ensure that ethical concerns regarding the specification of the project, like the ones stated above, are addressed.
Their responsibility is to ensure that the project is not used for any illicit purposes, or distributed unintentionally to malicious actors. 
Both the customers and restaurants using the service are responsible for reporting bugs or problems so they can be addressed. 
The restaurants are also responsible for ensuring their data is up-to-date and ensuring their food follows regulations.
\newline
\newline
Drone delivery is still not a popular method of delivery, with the first proper drone delivery service being rolled out this year\supercite{drone-delivery-royal-mail}. 
The emergent nature of drone use means that there has not been enough time for the associated ethical concerns to be fully examined and explored. 
Therefore, while some of the system's potential for harm has been explored in this essay, there are still many possibilities that could not be considered. 
Therefore, it will never be possible to assert that this system is definitively safe. 
However, if the recommended precautions are undertaken, this potential for harm will be reduced as much as possible, reducing the likelihood of unintentional harm. 
\printbibliography
\end{document}
